\documentclass[11pt,a4paper,titlepage]{report}
\usepackage[utf8]{inputenc}
\usepackage[english]{babel}
\usepackage{amsmath}
\usepackage{amsfonts}
\usepackage{amssymb}
\usepackage[left=2cm,right=2cm,top=2cm,bottom=2cm]{geometry}

\author{Adrian Wälchli}
\title{Bachelor Project Journal}

\begin{document}

\maketitle
\begin{abstract}
This report presents an overview of my bachelor thesis. We discuss several approaches to problems, experiments, ideas and evaluate results. This document will be extended over time as the project continues.
\end{abstract}

\tableofcontents

\section{Related work}
The basis of this project are the papers from \cite{WETZ_TOMO, WETZ_TENS}. Additional papers used for the work are: Light Field Rendering by Marc Levoy and Pat Hanrahan, Fourier Slice Photography by Ren Ng, Light Field Photography with a Hand-held Plenoptic Camera by Ren Ng et al. 

\section{Types of light fields}
In this project, I encountered two types of 4D light fields that are captured using camera grids. The most common way of acquiring a light field is to capture a scene with a 2D-grid of cameras where the optical axes of the cameras are orthogonal to the camera grid. Since the look-at-point of each camera is different, this setup will result in a shift in the images formed on the sensors. An alternative way to capture the scene is to fix the look-at-point for every camera, preferably at the origin of the  scene. This is the type of light fields primarily used in the paper from \cite{WETZ_TOMO}. 
\\
In addition, the images can be optained using either perspective or orthographic projections. We can also use sheared projections as mentioned in \cite[p.~4]{LEVO_LFREN}.

\section{A first implementation}
In a first step, I tried to implement the 

\section{Moving to light fields of type 1}

\section{Converting a light field}


\bibliographystyle{alpha}
\bibliography{lit}

\end{document}


