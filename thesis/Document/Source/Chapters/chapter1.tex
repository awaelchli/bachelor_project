\chapter{Introduction}
\label{chp:introduction}
%TODO: Figures: Lytro camera, Stanford camera array, Tensor Display etc.

Over the last few years, devices capable of displaying 3D content have shown to become increasingly popular.
Most of the moviegoers have long accustomed to the variety of movies releasing in 3D every year, and with affordable television screens on the market, movies with the extra dimension can be enjoyed in the living room.
Current graphics processors are powerful enough to bring the 3D experience to the video game consumer, allowing for a higher immersion into the virtual world.

There are two main categories of such displays, stereoscopic- and true 3D displays.

\subsection*{Stereoscopic Displays}

Stereoscopic displays are based on the principles of binocular vision.
The objective is to provide two distinct images to the human visual system, one for each eye, presenting the content from two slightly different perspectives.
The disparities between the two images translate to depth cues in the human brain and allow for depth perception.
The pair of images presented to the eyes remains constant when the viewer moves in front of the device. 
This effect distinguishes stereoscopic displays from 3D displays.
Modern technologies include head-mounted displays, polarization systems, active shutter systems and autostereoscopy.
Although not as comfortable to wear, head-mounted displays have separate high resolution screens for each eye allowing for a high degree of immersion.
Polarization screens show the image pair superimposed with different polarization of the light, which is separated again by different polarization filters in the right and left side of the viewers eyeglasses.
Active shutter systems use special eyeglasses that alternately block the light for one eye, letting the opposite eye see the corresponding image on the synchronized screen.
Autostereoscopic displays present stereo content to the viewer without the need of special glasses. 
The technology is based on a lenticular lens or parallax barriers, which requires the viewer to be in a fixed and predefined position. 

\subsection*{3D Displays}

Real 3D displays show, to a certain extent, the full 3D information to the observer.

Present technologies include volumetric displays, holography, integral imaging and compressive light field displays.

\section{Related Work}
