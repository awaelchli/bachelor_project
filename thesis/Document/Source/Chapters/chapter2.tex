\chapter{Capturing a Light Field}

\section{The Light Field and It's Properties}

The plenoptic function, as introduced by~\cite{AdelsonBergen}, is a 7D function that describes the intensity of light for every frequency, along every light ray in space, at any time. 
It is defined as
\begin{align*}
	P \colon \mathbb{R}^3 \times \left[0, 2 \pi \right) \times \left[ 0, \pi \right] \times \mathbb{R}^2 & \to \mathbb{R}^+ \\
	\left(x, y, z, \theta, \phi, t, \lambda \right) & \mapsto P\left(x, y, z, \theta, \phi, t, \lambda \right), 
\end{align*}
where the parameters $\left(x, y, z\right)$ are the coordinates of a point in 3D space and the angles $\left(\theta, \phi \right)$ describe the direction of an incoming light ray at time $t$. 
The light's intensity is given for every wavelength $\lambda$ and thus, the plenoptic function not only captures the visible frequency spectrum but all electromagnetic waves. 
A commonly used measure for light is the radiance, which is obtained from P by integrating over all wavelengths: 
$R\left(x, y, z, \theta, \phi, t\right) = \int_{\mathbb{R}} \! P\left(x, y, z, \theta, \phi, t, \lambda \right) \, \mathrm{d} \lambda$.

In practice, it is impossible to acquire all the data needed to model the 7D plenoptic function and hence it is reasonable to consider only a subset of the parameters. 
Dropping the time parameter $t$ in $R\left( x, y, z, \theta, \phi, t \right) $ yields a 5D function for the radiance in a static scene. 
As described by~\cite{LightFieldRendering}, this five dimensional representation can further be reduced to four dimensions in the following way. 
The radiance along a line is constant in free space and so, the 5D plenoptic function holds redundant information for the points on this line. 
Ignoring this redundancy leads to the equivalent 4D parametrization of the ray space. 
\cite{LightFieldRendering} propose a parametrization by two parallel planes, as seen in figure~\ref{fig:LightFieldParametrization}, where the coordinates of the lines (rays) are given by the intersections with the two planes.
The 4D light field $L(u, v, x, y)$ is therefore defined as the radiance along the line intersecting the two planes at coordinates $(u, v)$ and $(x, y)$.



\begin{figure}
	\centering
	\includegraphics[height=5cm]{placeholder}
	\caption{Parametrization of the light field with two planes.}
	\label{fig:LightFieldParametrization}
\end{figure}


%\begin{center}
%	
%	\tdplotsetmaincoords{70}{130}
%	
%	\begin{tikzpicture}[tdplot_main_coords, scale = 3]
%		
%		\filldraw[draw = black, fill = white] (-1, -1, -1) -- (-1, -1, 1) -- (-1, 1, 1) -- (-1, 1, -1) -- cycle;
%		\filldraw[draw = black, fill = white] (1, -1, -1) -- (1, -1, 1) -- (1, 1, 1) -- (1, 1, -1) -- cycle;
%	
%		
%		\draw[->] (0, 0, 0) -- (1, 0, 0) node[below left]{$x$};
%		\draw[->] (0, 0, 0) -- (0, 1, 0) node[below right]{$y$};
%		\draw[->] (0, 0, 0) -- (0, 0, 1) node[above]{$z$};
%		
%	\end{tikzpicture}
%	
%\end{center}
