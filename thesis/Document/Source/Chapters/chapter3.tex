\chapter{Light Field Tomography}

\section{A Model for Attenuation}
%TODO: Derivation according to Wetzstein
% Explain linearity in log domain
% Mention other tomographic projection types
% SART and ART solvers
% Make a chapter?
% Explain "intensity of ray" and unit

The light field display is modeled by a volumetric attenuator $\mu(x, y, z)$ that attenuates the light traveling through its material.
According to the Beer-Lambert law, the intensity of a light ray $\mathcal{R} \subset \mathbb{R}^3$ passing through the material decreases exponentially over distance:
\begin{equation}\label{eq:beer_lambert_law}
I = I_0 e^{-\int_\mathcal{R} \mu(r) \, \mathrm{d}r }.
\end{equation}
The incident intensity $I_0$ is the intensity of the ray before it enters the attenuator.
Equation~\ref{eq:beer_lambert_law} can be rewritten into 
\begin{equation}\label{eq:log_beer_lambert_law}
\bar{I} \coloneqq \log \left( \frac{I}{I_0} \right) = -\int_\mathcal{R} \mu(r) \, \mathrm{d}r.
\end{equation} 
Now, let the attenuator $\mu(x, y, z)$ be a cubic slab of height $d$ in Z-direction and let $L(u, v, s, t)$ be the two-plane parameterization of the light field such that the $(s, t)$-plane coincides with the $(x, y)$-plane of the attenuator and the $(u, v)$-plane is at distance $d$.
The set of points describing the ray defined by the coordinates $(u, v, s, t)$ is
\begin{equation}
\mathcal{R} = \left\{ \lambda a + b 
\mathrel{\bigg|} a = 
\begin{pmatrix}
u - s \\ 
v - t \\ 
d
\end{pmatrix}, 
b = 
\begin{pmatrix}
s \\ 
t \\ 
0
\end{pmatrix},
\lambda \in \mathbb{R} 
\right\}.
\end{equation}
A point $p = (x, y, z)^T$ is part of the ray $\mathcal{R}$ if and only if
\begin{align}
& \exists \lambda \in \mathbb{R} : p = \lambda a + b & & \iff & & a \times (p - b) = 0, 
\end{align} 
where $\times$ denotes the cross product.
Now, $I$ can be replaced with the light field $L$ and the right hand side of equation~\ref{eq:log_beer_lambert_law} can be written as an integral over~$\mathbb{R}^3$:
\begin{equation}\label{eq:log_lightfield_and_radon_transform}
\bar{L}(u, v, s, t) = 	%\int \limits_{-\infty}^{\infty} \int \limits_{-\infty}^{\infty} \int \limits_{-\infty}^{\infty}
-\int_{\mathbb{R}^3}
\mu(p) \delta ( a \times (p - b) ) \, 
\mathrm{d}p.
\end{equation}
Here, $\delta$ denotes the Dirac delta function on $\mathbb{R}^3$ and $\mu$ is zero outside the boundaries of the slab. 
This means that the integrand is only non-zero for points on the ray with coordinates $(u, v, s, t)$.

Combining equation~\ref{eq:beer_lambert_law} and~\ref{eq:log_lightfield_and_radon_transform} gives the light field emitted by the attenuator.
The goal is to produce such an attenuation display that emits a given target light field.

In computed tomography, the \textbf{Radon transform} of a real valued and compactly supported, continuous function $f(x, y)$ on $\mathbb{R}^2$ is defined as
\begin{equation}
p(\rho, \theta) = 	\int \limits_{-\infty}^{\infty} 
\int \limits_{-\infty}^{\infty}
f(x, y) \delta (x \cos \theta - y \sin \theta - \rho) \, 
\mathrm{d}x \,
\mathrm{d}y,
\end{equation}
where $(\rho, \theta) \in \mathbb{R} \times \left(- \frac{\pi}{2}, \frac{\pi}{2}\right)$ defines a ray as shown in figure~\ref{fig:radon_transform_2D_sketch}.
Because the Radon transform is essentially a line integral, it can be generalized to three or more dimensions.
\begin{figure}[tb]
	\centering
	\documentclass{standalone}
\usepackage{tikz}
\usetikzlibrary{intersections}

\begin{document}
	
	\begin{tikzpicture}[scale = 0.5]
		
		\draw[->] (-5, 0) -- (5, 0);
		\draw[->] (0, -5) -- (0, 5);
		\node[right] at (5, 0) {$x$};
		\node[above] at (0, 5) {$y$};
		
		\draw plot[smooth cycle] coordinates {(-3, 1) (-2, 2.5) (0, 2) (1.5, 3.5) (2, 2) (2.5, 1) (2, 0) (2, -2) (0.5, -1.5) (-2, -3) (-2, -1)};
		
		\draw[<-] (-1.5, 7.5) -- (-7.5, -1.5); %node[below right] {$\rho$};
		
		\draw[name path = ray, ->] (4, -0.5) -- (-6.5, 6.5);
		\draw (0, 0) -- (-7.5, 5);
		
		\draw (0, 1.5) arc (90 : 180 - atan(2 / 3) : 1.5);
		\node at (-0.4, 0.8) {$\theta$};
		
		\draw[->] (0, 0) -- (1, 1.5) node[pos = 0.4, right] {$\rho$};
		
		\draw[name path = radon] plot[smooth] coordinates {(-6.5, 0) (-6.7, 0.5) (-6.5, 1.2) (-6, 2) (-6.5, 3) (-6, 4.5) (-5, 4.5) (-4, 5.5) (-2.5, 6)};
		
		\fill[name intersections={of= radon and ray, total=\t}]
		\foreach \s in {1,...,\t}{(intersection-\s) circle[radius = 0.1] node[rotate= 90 - atan(2 / 3), above right] {$p(\rho, \theta)$}};
		
		\node[rotate= 90 - atan(2 / 3), right] at (-1.5, 7.5) {$\rho$};
		
	\end{tikzpicture}
	
\end{document}
	\caption{The 2D Radon transform of the ray $(\rho, \theta)$ passing a material with density $f(x, y)$.}
	\label{fig:radon_transform_2D_sketch}
\end{figure}
Adapting the notation from the two-plane parameterization, the Radon transform of the attenuation map $\mu$ along ray $\mathcal{R}$ becomes
\begin{equation}
p(u, v, s, t) = 	\int \limits_{-\infty}^{\infty} 
\int \limits_{-\infty}^{\infty}
\int \limits_{-\infty}^{\infty}
\mu(x, y, z) \delta \left(a \times \left((x, y, z)^T - b\right) \right) \, 
\mathrm{d}x \,
\mathrm{d}y \,
\mathrm{d}z, 
\end{equation}
which is equivalent to equation~\ref{eq:log_lightfield_and_radon_transform}.
This shows that
\begin{equation}\label{eq:log_light_field_negative_radon}
\bar{L}(u, v, s, t) = -p(u, v, s, t), 
\end{equation}
or with the words of~\cite{WetzsteinTomo}: \say{The logarithm of the emitted light field is equivalent to the negative Radon transform of the attenuation map.}

\section{Discrete Attenuation Layers}

The previous section introduced a continuously varying attenuation map to model the display.
\cite{WetzsteinTomo} propose to represent the attenuator with a set of $N$ two-dimensional layers, also called masks.

Let $L_{ijkl} = L(u(i), v(j), s(k), t(l))$ be the matrix of samples from the light field and for simplicity, let $m \coloneqq m(i, j, k, l)$ be a linear index of the 4D indices.
Equation \ref{eq:beer_lambert_law} suggests a per-ray constraint in the form
\begin{equation}\label{eq:transmittance_layers}
	L_m = L_0 \prod_{n=1}^{N} t^{(n)} (h(m, n)), 
\end{equation}
where $h(m, n)$ is the (discrete) 2D coordinate of the intersection of the \mbox{$m$-th} ray with the \mbox{$n$-th} layer, and $t^{(n)}(\xi)$ is the \textbf{transmittance} of layer $n$ at that coordinate.
Having a constraint for each ray, the goal is to solve for the transmittance~$t$.
However, the system of equations in~\ref{eq:transmittance_layers} is non-linear and cannot directly be solved.
One can obtain a linear system of equations by taking the logarithm in~\ref{eq:transmittance_layers}:
\begin{equation}\label{eq:discrete_log_light_field_negative_radon}
	\bar{L}_m 	=	\sum_{n = 1}^{N}
					\log \left( t^{(n)} (h(m, n)) \right) 
				= 	-\sum_{n = 1}^{N} a^{(n)} (h(m, n)) 
				= -P_m \alpha.
\end{equation}
Here, $a^{(n)} \coloneqq -\log t^{(n)}$ denotes the \textbf{absorbance} of layer $n$. 
This relation between transmittance and absorbance also directly follows from the Beer-Lambert law.
Here, $P_m = \left( P_m^{(1)}, \dots, P_m^{(N)} \right)$ is a binary row vector, encoding the intersection of the ray with the pixels on each layer.
The unknown absorbance is represented by the column vector $\alpha = \left( \alpha^{(1)}, \dots, \alpha^{(N)} \right)^T$.
Each $\alpha^{(i)}$ is just a flattened representation of the absorbance matrix $a^{(i)}$.
Note that equation~\ref{eq:discrete_log_light_field_negative_radon} is the equivalent of the continuous version in~\ref{eq:log_light_field_negative_radon}, since $P_m$ encodes the Radon transform.
Finally, the above equations indexed by $m$ can be combined into one large linear system $P \alpha = -\bar{L}$.

In most cases, $P$ is not a square matrix and the system can become overdetermined, which means that it has no solution in general.
However, it is still possible to find values for $\alpha$ such that the error $\lVert P \alpha + \bar{L} \rVert$ is small. 
Thus, the objective becomes
\begin{equation} \label{eq:minimize_norm}
	\begin{aligned}
		& \underset{x}{\text{argmin}} & & \lVert P \alpha + \bar{L} \rVert \\
		& \text{subject to} & & 0 \leq \alpha \leq \infty.
	\end{aligned}
\end{equation}
Finally, when optimal values $\alpha$ are found, the transmittance used to fabricate the layers is obtained by calculating $e^{-\alpha}$.

\section{Iterative Reconstruction}


